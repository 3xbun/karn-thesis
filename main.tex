\documentclass[12pt,twoside]{report}

\usepackage{apacite}
\usepackage{indentfirst}
\usepackage{fontspec}
\usepackage{xunicode}
\usepackage{xltxtra}
\usepackage{titlesec}
\usepackage[a4paper,left=25mm,right=25mm,top=25mm,bottom=25mm]{geometry}

\defaultfontfeatures{Mapping=tex-text}
\setmainfont{THSarabunNew}[
    Path=./Resources/Fonts/,
    Extension=.ttf,
    BoldFont=* Bold,
    ItalicFont=* Italic,
    BoldItalicFont=* BoldItalic
]
\XeTeXlinebreaklocale `TH`
\XeTeXlinebreakskip = 0pt plus 1pt

\newcommand{\quotes}[1]{``#1''}

\renewcommand{\thesection}{บทที่ \arabic{section}}
\renewcommand{\thesubsection}{\arabic{section}.\arabic{subsection}}

\titleformat{\section}[display]{\bfseries\centering\Large}{\thesection}{0em}{\Large}{}

\begin{document}
\section{บทนำ}

\subsection{ความเป็นมาของปัญหา}
โรคมาลาเรียเป็นโรคติดต่อ มียุงก้นปล่องเป็นพาหะเกิดจากเชื้อ Plasmodium ซึ่งเป็นสัตว์เซลเดียวอยู่ใน class Sporozoa มาลาเรียเป็นปัญหาสาธารณสุขที่สำคัญมาก ประชากรร้อยละ 36 ของประชากรจากกว่า 90 ประเทศทั่วโลกอาศัยอยู่ในบริเวณที่มีการแพร่กระจายของโรคมาลาเรีย สำหรับประเทศไทยมาลาเรียยังคงเป็นปัญหาสาธารณสุขที่สำคัญเช่นกัน แม้ว่าโรคนี้จะมีอัตราป่วยและอัตราตายลดลง แต่ตั้งแต่ปี พ.ศ. 2539 เป็นต้นมา อัตราการป่วยและอัตราการตายมีแนวโน้มเพิ่มสูงขึ้น

โรคมาลาเรียเป็นปัญหาสุขภาพที่เกิดขึ้นทั่วโลกตั้งแต่อดีตจนถึงปัจจุบันอย่างต่อเนื่อง โดยในแต่ละปีมีผู้ติดเชื้อมาลาเรียกว่า 500 ล้านคนทั่วโลก ซึ่งส่วนใหญ่พบในกลุ่มประชากรที่อาศัยอยู่ในประเทศเขตร้อนของโลก ได้แก่ประเทศในทวีปแอฟริกา อเมริกากลาง อเมริกาใต้ และในเอเชียตะวันออกเฉียงใต้ โดยในปี พ.ศ. 2563 มีการบันทึกผู้ติดเชื้อไว้มากถึง 241 ล้านคน และเสียชีวิตกว่า 6 แสนคน คิดเป็น 24.89 ต่อแสนประชากร สำหรับในประเทศไทยจากสถานการณ์ของโรคตั้งแต่ปี พ.ศ. 2563 จนถึงปี พ.ศ. 2565 พบว่าประเทศไทยมีรายงานผู้ป่วยโรคมาลาเรียสูงมากขึ้นอย่างมีนัยสำคัญ โดยในปี พ.ศ. 2563 พบผู้ติดเชื้อทังหมด 3,945 ราย ปี พ.ศ. 2564 3,266 ราย และปี พ.ศ. 2565 10,156 ราย โดยที่จังหวัดตาก จำนวนผู้ติดเชื้อยังเป็นไปในทิศทางเดียวกันกับจำนวนผู้ติดเชื้อในประเทศ และยังเป็นจังหวัดที่พบผู้ติดเชื้อมากที่สุด โดยในปี พ.ศ. 2563 พบผู้ติดเชื้อ 1,089 ราย ปี พ.ศ. 2564 1,132 ราย และในปี พ.ศ. 2565 จังหวัดตากพบผู้ติดเชื้อ 6,389 ราย  โดยในปี พ.ศ. 2565 นี้เองที่มีผู้ติดเชื้อคิดเป็นร้อยละ 62.91 ของจำนวนผู้ป่วยทั้งประเทศ และจะเห็นได้ว่าตั้งแต่ปี พ.ศ. 2563 ถึง พ.ศ. 2565 จังหวัดตากเป็นจังหวัดที่พบผู้ป่วยมาลาเรียเป็นอันดับหนึ่งในประเทศ และมีแนวโน้มที่เพิ่มขึ้นมากทุกปี โดยในปี พ.ศ. 2566 ตั้งแต่เดือนมกราคมจนถึงเดือนมิถุนายน พบผู้ติดเชื้อไปแล้วกว่า 3,758 ราย โดยเพิ่มขึ้นจาก 2 ไตรมาสแรกของปีที่แล้วกว่าร้อยละ 60 จังหวัดตาก โดยอำเภอที่พบผู้ติดเชื้อมากที่สุดในปี พ.ศ.2565 ได้แก่ อำเภอท่าสองยาง จำนวนผู้ติดเชื้อ 2,518 ราย อำเภออุ้มผาง จำนวน 1,662 ราย และอำเภอพบพระ จำนวน 1,137 ราย ตามลำดับ โดยในโรงเรียนประจำในเขตพื้นที่ตำบลแม่ท้อมีจำนวนเด็กนักเรียนที่มาจาก 3 ตำบลดังกล่าวในปีการศึกษา 2566 จำนวน 529 ราย

วิทยานิพนธ์ชิ้นนี้เกิดขึ้นจากการมองว่าการมีความรู้ด้านสาธารณสุขจะส่งผลต่อพฤติกรรมด้านสุขภาพ รวมไปถึงการตัดสินใจต่าง ๆ ที่จะส่งผลต่อสุขภาพของบุคคลนั้น ๆ และจากการศึกษาเพิ่มเติมพบว่าการเสริมสร้างความรู้ในเรื่องโรคมาลาเรียจะส่งผลโดยตรงต่อการแพร่กระจายของเชื้อ อย่างไรก็ตามการศึกษาส่วนใหญ่มักมุ่งเน้นไปที่ชุมชน และไม่ได้ให้ความสำคัญต่อสังคมในโรงเรียนประจำ ซึ่งสถานศึกษาเช่นนี้มีคุณลักษณะเฉพาะตัว และมีความท้าทายเป็นอย่างมาก

การมุ่งเน้นไปที่โรงเรียนประจำจะทำให้การทำวิทยานิพนธ์ชิ้นนี้ช่วยสำรวจความรู้ที่ขาดหายไปจากนักเรียนภายในโรงเรียน รวมไปถึงช่วยปรับเปลี่ยนพฤติกรรมต่างที่ส่งผลให้นักเรียนเหล่านั้นอาจได้รับเชื้อมาลาเรียได้หากขาดความรู้เหล่านี้เพื่อเป็นแนวทางในการหาหัวข้อในการให้ความรู้ในโอกาสหน้า และนักเรียนที่อาศัยอยู่ในโรงเรียนประจำจะเป็นทรัพยากรในการทำวิทยานิพนธ์ที่ดีมากกว่าโรงเรียนทั่วไป เนื่องจากการที่ต้องอาศัยอยู่รวมกัน จะช่วยลดปัจจัยด้านที่อยู่อาศัย และปัจจัยด้านครอบลงไปได้ และสามารถมุ่งเน้นไปที่ความรู้ได้โดยตรง

ดังนั้นวิทยานิพนธ์เรื่อง xxx จึงจะเป็นส่วนสำคัญ และมีศักยภาพที่ดีในการสำรวจและปรับปรุงภูมิทัศน์ด้านสุขภาพในโรงเรียน และมีความหวังอย่างมากว่าจะส่งไปถึงชุมชนที่นักเรียนอาศัยอยู่อีกด้วย โดยการที่ผู้จัดทำวิทยานิพนธ์สามารถระบุช่องโหว่ของความรู้ และสามารถส่งต่อข้อมูลเพื่อสร้างการออกแบบการเรียนรู้ของนักเรียนในโรงเรียนนี้ได้ และจะสามารถสร้างวัฒนธรรมการป้องการเกิดโรคมาลาเรียและลดการแพร่กระจายของโรคได้จากการมีข้อมูลเหล่านี้ สุดท้ายแล้วผู้จัดทำวิทยานิพนธ์หวังว่างานวิทยานิพนธ์ชิ้นนี้จะสามารถช่วยให้นักเรียนในโรงเรียนประจำแห่งนี้สามารถตระหนักรู้ได้ถึงช่องว่างของความรู้ของตนเอง และเป็นพลังในการผลักดันให้นักเรียนค้นคว้าพยายามปิดช่องว่างดังกล่าว และเป็นตัวแทนของการเปลี่ยนแปลงในการสร้างสังคมปลอดโรคมาลาเรีย

\subsection{วัตถุประสงค์ของการวิจัย}
\begin{enumerate}
    \item เพื่อประเมินระดับความรู้ของนักเรียนในโรงเรียนเกี่ยวกับโรคมาลาเรีย
    \item เพื่อระบุช่องว่างความรู้ และความเข้าใจผิดที่มีอยู่ในนักเรียนเกี่ยวกับการป้องกันโรคมาลาเรีย
    \item เพื่อตรวจสอบความสัมพันธ์ระหว่างความรู้ของนักเรียนเกี่ยวกับโรคมาลาเรีย และการเลือกพฤติกรรมในการป้องกันโรคมาลาเรีย
    \item เพื่อสำรวจศักยภาพของนักเรียนในฐานะตัวแทนของการเปลี่ยนแปลงเพื่อการส่งเสริมการป้องกันและการรับรู้โรคมาลาเรียภายในครอบครัวและชุมชน
\end{enumerate}

\subsection{ขอบเขตของงานวิจัย}
\begin{enumerate}
    \item ประชากร
          เด็กนักเรียนโรงเรียนประจำในเขตพื้นที่ตำบลแม่ท้อ จำนวน
    \item กลุ่มตัวอย่าง
    \item ระยะเวลาในการทดลอง
\end{enumerate}

\subsection{กรอบแนวคิดการวิจัย}

\subsection{คำสำคัญของการวิจัย}
\begin{enumerate}
    \item โรคมาลาเรีย
    \item โรงเรียนประจำ xxx
    \item ความรู้
    \item พฤติกรรม
\end{enumerate}

\subsection{ประโยชน์ที่คาดว่าจะได้รับ}
\begin{enumerate}
    \item นักเรียนมีความรู้ด้านสุขภาพที่ดีขึ้น

          ปรับปรุงความรู้ด้านสุขภาพของนักเรียน นักเรียนจะได้ดรับข้อมูลที่ถูกต้องเกี่ยวกับโรค การแพร่เชื้อ และมาตรการป้องกันที่มีประสิทธิภาพ โดยการประเมินความเข้าใจเกี่ยวกับโรคมาลาเรีย
    \item นักเรียนมีความรู้ที่จะส่งผลต่อการปรับเปลี่ยนพฤติกรรม

          โน้มน้าวพฤติกรรมของนักเรียนในทางบวก โดยการระบุช่องว่างความรู้ และความเข้าใจผิด นักเรียนจะได้รับความรู้ที่จะส่งผลต่อการปรับเปลี่ยนพฤติกรรมของนักเรียน การเปลี่ยนแปลงพฤติกรรมนี้จะช่วยลดความเสี่ยงของการแพร่เชื้อมาลาเรียภายในชุมชนของโรงเรียน
    \item นักเรียนที่มีความรู้ที่ถูกต้องจะส่งผลกระทบต่อชุมชนไปในทางที่ดี

          นักเรียนจะสร้างแรงกระเพื่อมในชุมชนที่มากขึ้น นักเรียนที่มีความรู้และมีพลังในการป้องกันโรคมาลาเรียจะสามารถชักจูงครอบครัว และเพื่อน ๆ ให้นำมาตรการป้องกันมาใช้ และพยายามร่วมกันนำไปสู่การลดจำนวนผู้ป่วยโรคมาลาเรียลงได้อย่างมีนับสำคัญ และปรับปรุงสุขภาพโดยรวมของชุมชนที่นักเรียนอาศัยอยู่
    \item ประโยชน์ด้านสาธารณสุขในระยะยาว

          นักเรียนในฐานะตัวแทนของการเปลี่ยนจะนำความรู้และพฤติกรรมการป้องกันของตนไปใช้นอกเวลาเรียน สงเสริมการลดการแพร่กระจ่ายของโรคมาลาเรียอย่างยั่งบยืน และมีส่วนร่วมในการสร้างสังคมที่ปราศจากโรคมาลาเรีย
    \item 	ประหยัดค่าใช้จ่าย

          มาตรการป้องกันโรคมาลาเรียมีประสิทธิภาพที่จะส่งผลให้ครอบครัวของนักเรียนประหยัดค่าใช้จ่ายสำหรับการรักษาพยาบาล และการลดอุบัติการณ์ของโรคมาลาเรียจะทำให้ค่ารักษาพยาบาลที่เกี่ยวข้องกับการวินัจฉัย การรักษา และค่าใช้จ่ายภายในโรงพยบาบาลลดลง นอกจากนี้ภาระทางเศรษฐกิจของครอบครับและชุมชนเนื่องจากการสูญเสียบุคลากรในครอบครัวและชุมชนไปด้วยการติดเชื้อมาลาเรียก็จะลดลงอย่างมีนัยสำคัญ
\end{enumerate}
\newpage

\section{Test Ref}
Citation 1 \cite{test01}

\section{ยากกกกกกกก}
ทดสอบ sync github

\bibliographystyle{apacite}
\bibliography{biblography}
\end{document}
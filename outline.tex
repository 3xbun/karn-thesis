\begin{titlepage}

  \begin{center}
    \includegraphics[width=0.3\textwidth]{nu-logo}

    \textbf{โครงร่างวิทยานิพนธ์}
  \end{center}


  \begin{table}[h]
    \centering
    \begin{tabular}{p{4cm} p{11cm}}
      \textbf{ชื่อเรื่องวิทยานิพนธ์}        & ความสัมพันธ์ระหว่างความรอบรู้ด้านสุขภาพกับพฤติกรรมการป้องกันโรคมาลาเรียของเด็กนักเรียนโรงเรียนประจำในเขตตำบลแม่ท้อ อ.เมือง จ.ตาก                    \\

                                     & The Relationship between Health Literacy and Malaria Prevention Behaviours among Students in Boarding School at Mae Tho, Tak. \\

      \textbf{ชื่อนิสิต}                 &
      นางสาวณัฐกานต์ เครือมี                                                                                                                                             \\

      \textbf{รหัสประจำตัว}             & 66060985                                                                                                                      \\

      \textbf{ปริญญา}                 & สาธารณสุขศาสตรมหาบัณฑิต                                                                                                          \\

      \textbf{สาขาวิชา}               & -                                                                                                                             \\

      \textbf{ประธานที่ปรึกษาวิทยานิพนธ์}  & [ตำแหน่งทางวิชาการ ชื่อ ชื่อสกุล]                                                                                                     \\

      \textbf{กรรมการที่ปรึกษาวิทยานิพนธ์} & [ตำแหน่งทางวิชาการ ชื่อ ชื่อสกุล]                                                                                                     \\

      \textbf{กรรมการที่ปรึกษาวิทยานิพนธ์} & [ตำแหน่งทางวิชาการ ชื่อ ชื่อสกุล]                                                                                                     \\

      \textbf{ปีการศึกษา}              & 2566
    \end{tabular}
  \end{table}

  \newpage

  \centering\textbf{โครงร่างวิทยานิพนธ์}


  \raggedright\textbf{เรื่อง} ความสัมพันธ์ระหว่างความรอบรู้ด้านสุขภาพกับพฤติกรรมการป้องกันโรคมาลาเรียของเด็กนักเรียนโรงเรียนประจำในเขตตำบลแม่ท้อ อ.เมือง จ.ตาก

  \begin{enumerate}
    \justifying
    \item \textbf{ความเป็นมาและความสำคัญ}

          โรคมาลาเรียเป็นโรคติดเชื้อโปรโตซัว ในกลุ่มพลาสโมเดียม (Plasmodium spp.) เป็นสัตว์เซลล์เดียวอยู่ใน class Sporozoa  ซึ่งติดต่อสู่คนโดยมียุงก้นปล่อง (Anopheles spp.) เป็นพาหะ โรคมาลาเรียส่วนใหญ่จะพบมากในประเทศเขตร้อนชื้น และจะพบการระบาดในช่วงฤดูฝน โดยยุงก้นปล่องเพศเมีย จะวางไข่ในแหล่งน้ำตามธรรมชาติ ไข่จะฟักเป็นลูกน้ำยุงภายใน 2 – 3 วัน และมีระยะเวลาในการเป็นลูกน้ำอีก 9 – 12 วัน ก่อนที่จะกลายเป็นยุงตัวเต็มวัย โดยยุงก้นปล่องเพศเมียเท่านั้นที่ดูดเลือดคนและสัตว์ที่สามารถนำเชื้อมาลาเรียได้ ผู้ที่รับเชื้อมาลาเรียไปแล้ว ส่วนใหญ่จะมีระยะฟักตัวของโรคประมาณ 10 – 14 วัน อาการสำคัญของโรคไข้มาลาเรีย คือ ไข้ หนาวสั่น ปวดศีรษะ ปวดกล้ามเนื้อ บางรายที่อาการรุนแรงอาจมีภาวะแทรกซ้อน เช่น ตับวาย ไตวาย ไข้มาลาเรียขึ้นสมอง ทำให้เสียชีวิตได้หากไม่ได้รับการรักษาที่ถูกต้องอย่างทันถ่วงที (แนวทางเวชปฏิบัติในการรักษาผู้ป่วยโรคไข้มาลาเรีย ประเทศไทย ปี2564) (แนวทางการให้สุขศึกษาเรื่องการกินยาสำหรับผู้ป่วยโรคไข้มาลาเรียที่ไม่มีภาวะแทรกซ้อนในสถานบริการสาธารณสุข) โรคมาลาเรียเป็นปัญหาสาธารณสุขที่สำคัญมาก เนื่องจากผลกระทบของโรคมาลาเรียก่อให้เกิดความสูญเสียภาวะสุขภาพจากการเจ็บป่วยทั้งด้านร่างกาย  และส่งผลกระทบรายได้ทางเศรษฐกิจ ประชากรร้อยละ 36 ของประชากรจากกว่า 90 ประเทศทั่วโลกอาศัยอยู่ในบริเวณที่มีการแพร่กระจายของโรคมาลาเรีย (World Health Organization,2023)

          จากการรายงานขององค์การอนามัยโลก ในแต่ละปีมีผู้ติดเชื้อมาลาเรียกว่า 500 ล้านคนทั่วโลก ซึ่งส่วนใหญ่พบในกลุ่มประชากรที่อาศัยอยู่ในประเทศเขตร้อนของโลก ได้แก่ประเทศในทวีปแอฟริกา อเมริกากลาง อเมริกาใต้ และในเอเชียตะวันออกเฉียงใต้ โดยในปี พ.ศ. 2564  มีการบันทึกผู้ติดเชื้อไว้มากถึง 241 ล้านคน และเสียชีวิตกว่า 6 แสนคน คิดเป็น 24.89 ต่อแสนประชากร (World Health Organization,2023

          ภูมิศาสตร์ของประเทศไทยนั้นมีเขตติดต่อกับประเทศประเทศเพื่อนบ้าน คือ เมียนมาร์ กัมพูชา ลาว และมาเลเซีย ซึ่งการแพร่เชื้อของโรคมาลาเรียจะมีสูงมากในจังหวัดชายแดนที่มีป่าเขาเชื่อมต่อกับประเทศเพื่อนบ้าน ซึ่งมียุงพาหะอาศัยอยู่ จังหวัดและอำเภอชายแดนจึงเป็นพื้นที่เปราะบางต่อการแพร่โรคเนื่องจากมีสภาพภูมิประเทศ ภูมิอากาศ สิ่งแวดล้อมทางธรรมชาติและทางสังคมที่เอื้อต่อการแพร่ระบาดของเชื้อมาลาเรีย  (แนวทางเวชปฏิบัติในการรักษาผู้ป่วยโรคไข้มาลาเรีย ประเทศไทย ปี2564)

          สถานการณ์ของโรคมาลาเรียในประเทศไทยแม้ว่าโรคนี้จะมีอัตราป่วยและอัตราตายลดลง แต่ตั้งแต่ปี  พ.ศ. 2539 เป็นต้นมา แต่อัตราการป่วยและอัตราการตายมีแนวโน้มเพิ่มสูงขึ้น โดยในปี พ.ศ. 2563 พบผู้ติด เชื้อทั้งหมด 3,945 ราย ปี พ.ศ. 2564 3,266 ราย และปี พ.ศ. 2565 10,156 ราย  เพิ่มขึ้นจากปี พ.ศ. 2564 จำนวน 6,890 ราย (malaria.ddc.moph.go.th)

          จังหวัดตากเป็นจังหวัดทางภาคเหนือ มีพื้นที่ชายแดนติดกับประเทศเมียนมาร์  และมีช่องทางเข้าออกจํานวนมาก และประชากรตามแนวชายแดนมีการเคลื่อนย้ายสูงทําให้มีการแพร่กระจายโรคตามแนวชายแดนโดยที่จังหวัดตาก พบจำนวนผู้ติดเชื้อที่เป็นไปในทิศทางเดียวกันกับจำนวนผู้ติดเชื้อในประเทศ และยังเป็นจังหวัดที่พบผู้ติดเชื้อมากที่สุด โดยในปี พ.ศ. 2563 พบผู้ติดเชื้อ 1,089 ราย ปี พ.ศ. 2564 1,132 ราย และในปี พ.ศ. 2565 จังหวัดตากพบผู้ติดเชื้อ 6,389 ราย  โดยในปี พ.ศ. 2565 นี้เองที่มีผู้ติดเชื้อคิดเป็นร้อยละ 62.91 ของจำนวนผู้ป่วยทั้งประเทศ และจะเห็นได้ว่าตั้งแต่ปี พ.ศ. 2563 ถึง พ.ศ. 2565 จังหวัดตากเป็นจังหวัดที่พบผู้ป่วยมาลาเรียเป็นอันดับหนึ่งในประเทศ และมีแนวโน้มที่เพิ่มขึ้นมากทุกปี โดยในปี พ.ศ. 2566 ตั้งแต่เดือนมกราคมจนถึงเดือนมิถุนายน พบผู้ติดเชื้อไปแล้วกว่า 3,758 ราย โดยเพิ่มขึ้นจาก 2 ไตรมาสแรกของปีที่แล้วกว่าร้อยละ 60 โดยอำเภอที่พบผู้ติดเชื้อมากที่สุดของจังหวัดตากในปี พ.ศ.2565 ได้แก่ อำเภอท่าสองยาง จำนวนผู้ติดเชื้อ 2,518 ราย อำเภออุ้มผาง จำนวน 1,662 ราย และอำเภอพบพระ จำนวน 1,137 ราย ตามลำดับ (สํานักโรคติดต่อนําโดยแมลง  กรมควบคุมโรค กระทรวงสาธารณสุข.  โครงการกําจัดโรคไข้  มาลาเรีย ระบบมาลาเรียออนไลน์ [อินเตอร์เน็ต]. [เข้าถึงเมื่อ  24  ก.ค.  2566].  เข้าถึงได้จาก:http://malaria.ddc.moph.go.th)

          จากการประเมินความรอบรู้ และพฤติกรรมด้านสุขภาพของประชาชน ส่วนใหญ่พบว่ามีความรู้ในเรื่องสุขภาพและพฤติกรรมสุขภาพอยู่ในระดับที่พอใช้ อย่างไรก็ตามระดับความรู้ด้านสุขภาพและพฤติกรรมสุขภาพยังไม่เพียงพอที่จะสนับสนุนการพัฒนาพฤติกรรมสุขภาพที่มีความยั่งยืนและส่งเสริมสุขภาพในระยะยาว สถานการณ์เช่นนี้อาจมีผลต่อสุขภาพโดยรวมของประชากรในอนาคต โดยเฉพาะหากประชากรส่วนใหญ่ในประเทศมีระดับความรู้ทางด้านสุขภาพที่ต่ำ อาจส่งผลกระทบต่อสภาวะสุขภาพทั่วไปของประชากร (กองสุขศึกษา, 2561) เมื่อประชาชนมีความรู้ทางด้านสุขภาพตั้งแต่ระยะเริ่มต้น จะช่วยลดความเสี่ยงที่อาจนำสู่สภาวะที่มีปัญหาดังกล่าว บุคคลที่มีความรู้ด้านสุขภาพต่ำจะมีผลกระทบต่อการนำข้อมูลไปใช้ การเข้ารับบริการดูแลสุขภาพ การรับมือกับโรคภัยไข้เจ็บด้วยตนเอง และการป้องกันโรค (DeWalt et al., 2004) จะสังเกตได้ว่าความรู้ทางด้านสุขภาพมีความสำคัญต่อการส่งเสริมพฤติกรรมในการป้องกันตนเอง (Nutbeam D, 2000) ตามคำนิยามขององค์การอนามัยโลก ความรู้ทางด้านสุขภาพหมายถึงกระบวนการความคิดเชิงปัญญาและทักษะทางสังคมที่กระตุ้นแรงบันดาลใจและความสามารถของบุคคลที่จะเข้าถึง เข้าใจ และนำข้อมูลและข่าวสารไปใช้เพื่อส่งเสริมและรักษาสุขภาพให้คงความเป็นอยู่อย่างมีคุณภาพตลอดเวลา (Smith B, Kwok C, Nutbeam D, 2011) บุคคลที่มีความรู้ในด้านสุขภาพจะมีความสามารถในการดูแลและรักษาสุขภาพของตนเองให้แข็งแรงในทุกช่วงวัย ซึ่งจะทำให้สามารถเข้าร่วมกิจกรรมและการทำงานได้อย่างมีพละกำลัง และป้องกันไม่ให้ถูกเล่นลอก และป้องกันการมีพฤติกรรมที่ไม่เหมาะสม บุคคลที่มีความรู้ด้านสุขภาพจะสามารถพึงพาตนเองเพื่อรักษาสุขภาพได้และยังสามารถมีบทบาทในการสร้างครอบครัวที่มีสุขภาพดี (ขวัญเมือง แก้วดำเกิง, 2562)

          ดังนั้น การสร้างความรู้ทางด้านสุขภาพในกลุ่มนักเรียน เป็นเวลาที่อยู่ในช่วงวัยกำลังเจริญเติบโตและการเรียนรู้ การดูแล และส่งเสริมสุขภาพนั้นจะเป็นการสร้างพื้นฐานสุขภาพที่แข็งแรง ซึ่งจะช่วยเพิ่มสมรรถภาพในการเรียนรู้และเติบโตเต็มพิกุล ร่างกายจะเจริญเติบโตและพัฒนาอย่างสม่ำเสมอ ซึ่งจะเป็นปัจจัยสำคัญในการเตรียมความพร้อมในการทำงานและเป็นแกนนำในการแชร์ข้อมูลเกี่ยวกับสุขภาพ อย่างไรก็ตาม หากไม่มีการดูแลสุขภาพหรือมีกิจกรรมที่มีความเสี่ยง อาจส่งผลให้เกิดการเจ็บป่วยซึ่งอาจเป็นอุปสรรคต่อกระบวนการเรียนรู้ และทำให้สูญเสียโอกาสหลายๆ ดังนั้น การใส่ใจสุขภาพเป็นสิ่งสำคัญ เพื่อประสิทธิภาพในการเรียนรู้และเติบโตเต็มพิกุล และเพื่อป้องกันผลเสียที่อาจเกิดขึ้นในอนาคต (กองสุขศึกษา, 2561)

          การมุ่งเน้นไปที่โรงเรียนประจำจะทำให้การทำวิทยานิพนธ์ชิ้นนี้ช่วยสำรวจความรู้ที่ขาดหายไปจากนักเรียนภายในโรงเรียน รวมไปถึงช่วยปรับเปลี่ยนพฤติกรรมต่างที่ส่งผลให้นักเรียนเหล่านั้นอาจได้รับเชื้อมาลาเรียได้หากขาดความรอบรู้เหล่านี้เพื่อเป็นแนวทางในการหาหัวข้อในการให้ความรู้ในโอกาสหน้า และนักเรียนที่อาศัยอยู่ในโรงเรียนประจำจะเป็นทรัพยากรในการทำวิทยานิพนธ์ที่ดีมากกว่าโรงเรียนทั่วไป เนื่องจากการที่ต้องอาศัยอยู่รวมกัน จะช่วยลดปัจจัยด้านที่อยู่อาศัย และปัจจัยด้านครอบครัวลงไปได้ และสามารถมุ่งเน้นไปที่ความรอบรู้ได้โดยตรง

          ดังนั้นวิทยานิพนธ์เรื่อง ความสัมพันธ์ระหว่างความรอบรู้ด้านสุขภาพกับพฤติกรรมการป้องกันโรคมาลาเรียของเด็กนักเรียนโรงเรียนประจำในเขตตำบลแม่ท้อ อ.เมือง จ.ตาก จึงจะเป็นส่วนสำคัญ และมีศักยภาพที่ดีในการสำรวจและปรับปรุงภูมิทัศน์ด้านสุขภาพในโรงเรียน และมีความหวังอย่างมากว่าจะส่งไปถึงชุมชนที่นักเรียนอาศัยอยู่อีกด้วย โดยการที่ผู้จัดทำวิทยานิพนธ์สามารถระบุช่องโหว่ของความรู้ และสามารถส่งต่อข้อมูลเพื่อสร้างการออกแบบการเรียนรู้ของนักเรียนในโรงเรียนนี้ได้ และจะสามารถสร้างวัฒนธรรมการป้องการเกิดโรคมาลาเรียและลดการแพร่กระจายของโรคได้จากการมีข้อมูลเหล่านี้ สุดท้ายแล้วผู้จัดทำวิทยานิพนธ์หวังว่างานวิทยานิพนธ์ชิ้นนี้จะสามารถช่วยให้นักเรียนในโรงเรียนประจำแห่งนี้สามารถตระหนักรู้ได้ถึงช่องว่างของความรู้ของตนเอง และเป็นพลังในการผลักดันให้นักเรียนค้นคว้าพยายามปิดช่องว่างดังกล่าว และเป็นตัวแทนของการเปลี่ยนแปลงในการสร้างสังคมปลอดโรคมาลาเรีย


          \item\textbf{จุดมุ่งหมายของการวิจัย}
          \begin{enumerate}
            \item เพื่อศึกษาความรอบรู้ด้านสุขภาพในการป้องกันโรคมาลเรียของเด็กนักเรียนโรงเรียนประจำในเขตตำบลแม่ท้อ อ.เมือง จ.ตาก
            \item เพื่อศึกษาพฤติกรรมการป้องกันโรคมาลเรียของเด็กนักเรียนโรงเรียนประจำในเขตตำบลแม่ท้อ อ.เมือง จ.ตาก
            \item เพื่อศึกษาความสัมพันธ์ระหว่างความรอบรู้ด้านสุขภาพในการป้องกันโรคมาลาเรียและพฤติกรรมการป้องกันโรคมาลาเรียของเด็กนักเรียนโรงเรียนประจำในเขตตำบลแม่ท้อ อ.เมือง จ.ตาก
          \end{enumerate}

          \item\textbf{ความสำคัญของการวิจัย}

          \item\textbf{ขอบเขตการวิจัย}

          \textbf{ขอบเขตด้านเนื้อหา}

          มุ่งเน้นการศึกษาความรอบรู้ด้านการป้องกันโรคมาลาเรีย ทั้ง 6 องค์ประกอบ ได้แก่
          \begin{enumerate}[itemindent=4em, label=\textnormal{\arabic*.}]
            \item การเข้าถึงข้อมูลสุขภาพและบริการสุขภาพ
            \item ความรู้ความเข้าใจ
            \item ทักษะการสื่อสาร
            \item ทักษะการจัดการตนเอง
            \item ทักษะการตัดสินใจ
            \item การรู้เท่าทั้นสื่อ
          \end{enumerate}

          \textbf{ขอบเขตด้านประชากร}

          ประชากรที่ใช้ในการวิจัยครั้งนี้ได้แก่ เด็กนักเรียนโรงเรียนประจำในเขตตำบลแม่ท้อ ในปีการศึกษา 2566 จำนวน 900 ราย

          กลุ่มตัวอย่าง เด็กที่เรียนในโรงเรียนประจำในเขตตำบลบ้านลานสาง จำนวน....

          \textbf{ขอบเขตด้านพื้นที่}

          การศึกษานี้ได้กำหนดการศึกษาในโรงเรียนประจำในเขตตำบลแม่ท้ออ.เมือง จ.ตาก

          \textbf{ขอบเขตด้านเวลาการวิจัย}

          สิงหาคม 2566-มิถุนายน 2567

          \textbf{ขอบแขตตัวแปรที่ศึกษา}

          ตัวแปรที่ใช้ในการศึกษา ประกอบด้วย ตัวแปรต้น และตัวแปรตาม มีรายละเอียด ดังนี้

          \begin{itemize}
            \item ตัวแปรอิสระ

                  \begin{itemize}
                    \item ปัจจัยส่วนบุคคลได้แก่ เพศ อายุ ระดับการศึกษา กลุ่มชาติพันธุ์ ประวัติการเจ็บป่วยโรคมาลาเรีย
                    \item ความรอบรู้ด้านสุขภาพในการป้องกันโรคมาลาเรียประกอบด้วย 6 องค์ประกอบ

                          \begin{enumerate}[label=\textnormal{\arabic*.}]
                            \item ความรอบรู้ด้านสุขภาพในการป้องกันโรคมาลาเรีย
                            \item การเข้าถึงข้อมูลสุขภาพและบริการสุขภาพด้านการป้องกันโรคมาลาเรีย
                            \item ทักษะการสื่อสารด้านการป้องกันโรคมาลเรีย
                            \item ทักษะการจัดการตนเองด้านการป้องกันโรคมาลาเรีย
                            \item ทักษะการตัดสินใจเกี่ยวกับการป้องกันโรคมาลาเรีย
                            \item การรู้เท่าทันสื่อเกี่ยวกับการป้องกันโรคมาลาเรีย
                          \end{enumerate}
                  \end{itemize}

            \item ตัวแปรตาม

                  พฤติกรรมการป้องกันโรคมาลาเรีย ได้แก่ พฤติกรรมการป้องกันโรคมาลาเรียเหมาะสม
          \end{itemize}

          \item\textbf{ข้อตกลงเบื้องต้น}

          \item\textbf{นิยามศัพท์เฉพาะ}
          \begin{enumerate}
            \item ความรอบรู้ด้านสุขภาพในการป้องกันโรคมาลาเรียหมายถึงความสามารถและทักษะในการเข้าถึงข้อมูลความรู้ความเข้าใจเพื่อวิเคราะห์ประเมินการปฏิบัติและการจัดการตนเองรวมทั้งสามารถชี้แนะเรื่องการป้องกันโรคมาลาเรียให้กับครอบครัวและชุมชนเพื่อสุขภาพที่ดีประกอบด้วยหกองค์ประกอบได้แก่
                  \begin{enumerate}
                    \item ความรู้ความเข้าใจในการป้องกันโรคมาลาเรียหมายถึงความเข้าใจที่ถูกต้องเกี่ยวกับแนวทางปฏิบัติด้านการป้องกันโรคมาลาเรีย
                    \item การเข้าถึงข้อมูลสุขภาพและบริการสุขภาพในการป้องกันโรคมาลาเรียหมายถึงการใช้ความสามารถในการเลือกแหล่งข้อมูลรู้วิธีในการค้นหาข้อมูลเกี่ยวกับการปฏิบัติตนด้านการป้องกันโรคมาลาเรียและตรวจสอบข้อมูลจากหลายแหล่งจนข้อมูลมีความน่าเชื่อถือ
                    \item ทักษะการสื่อสารสุขภาพในการป้องกันโรคมาลาเรียหมายถึงความสามารถในการสื่อสารโดยการพูดอ่านเขียนรวมทั้งสามารถสื่อสารและโน้มน้าวบุคคลอื่นเข้าใจและยอมรับข้อมูลเกี่ยวกับการปฏิบัติตนด้านการป้องกันโรคมาลาเรีย
                    \item ทักษะการจัดการตนเองในการป้องกันโรคมาลาเรียหมายถึงความสามารถในการกำหนดเป้าหมายวางแผนและปฏิบัติตามแผนการปฎิบัติด้านการป้องกันโรคมาลาเรียพร้อมทั้งมีการทบทวนวิธีการปฏิบัติตามเป้าหมายเพื่อนำมาปรับเปลี่ยนวิธีปฏิบัติตนที่ถูกต้อง
                    \item ทักษะการตัดสินใจในการป้องกันโรคมาลาเรียหมายถึงความสามารถในการกำหนดทางเลือกและ ปฏิเสธหลีกเลี่ยงหรือเลือกวิธีการปฏิบัติด้านการป้องกันโรคมาลาเรียโดยมีการใช้เหตุผลหรือวิเคราะห์ผลดี-ผลเสียเพื่อการปฏิเสธ /หลีกเลี่ยงพร้อมแสดงทางเลือกปฏิบัติที่ถูกต้อง
                    \item การรู้เท่าทันสื่อในการป้องกันโรคมาลาเรียหมายถึงความสามารถในการตรวจสอบความถูกต้องความน่าเชื่อถือของข้อมูลที่สื่อนำเสนอเกี่ยวกับการป้องกันโรคมาลาเรียและสามารถเปรียบเทียบวิธีการเลือกรับสื่อเพื่อเรียกเลี่ยงความเสี่ยงที่อาจเกิดขึ้นกับสุขภาพของตนเองและผู้อื่นรวมทั้งมีการประเมินข้อความสื่อเพื่อชี้แนะแนวทางให้กับชุมชนและสังคม
                  \end{enumerate}

            \item พฤติกรรมการป้องกันโรคมาลาเรียหมายถึงการปฏิบัติตนเพื่อป้องกันตนเองจากยุงก้นปล่องกัดหรือการปฏิบัติตนเพื่อไม่ให้ติดเชื้อหรือเจ็บป่วยด้วยโรคมาลาเรียและการจัดสภาพแวดล้อมเพื่อป้องกันการป่วยด้วยโรคมาลาเรียของ เด็กนักเรียนในโรงเรียนประจำในเขตตำบลแม่ท้ออำเภอเมืองจังหวัดตาก
            \item เด็กในโรงเรียนประจำ หมายถึงเด็กนักเรียนที่อาศัยอยู่ในโรงเรียนประจำ ในเขตตำบลแม่ท้ออำเภอเมือง จังหวัดตาก ในปีการศึกษา 2566
          \end{enumerate}

          \item\textbf{สมมุติฐานของการวิจัย}
    \item [] \quad ความรอบรู้ด้านสุขภาพในการป้องกันโรคมาลาเรียมีความสัมพันธ์กับพฤติกรรมการป้องกันโรคมาลาเรียของเด็กนักเรียนในโรงเรียนประจำในเขตตำบลแม่ท้อ อำเภอเมือง จังหวัดตาก


          \item\textbf{เอกสารและงานวิจัยที่เกี่ยวข้อง}

          \item\textbf{วิธีดำเนินการวิจัย}

          \begin{enumerate}

            \item ประชากรและกลุ่มตัวอย่าง

                  \lipsum[1-1]

            \item ตัวแปรที่ใช้ในการวิจัย

                  \lipsum[1-1]

            \item เครื่องมือและการพัฒนาเครื่องมือ

                  \lipsum[1-1]

            \item การเก็บรวบรวมข้อมูล

                  \lipsum[1-1]

            \item วิธีวิเคราะห์ข้อมูล

                  \lipsum[1-1]

          \end{enumerate}
          \item\textbf{แผนการดำเนินงาน}

          \lipsum[1-3]

          \item\textbf{เอกสารอ้างอิง/บรรณานุกรม}

          \lipsum[1-4]

  \end{enumerate}
\end{titlepage}